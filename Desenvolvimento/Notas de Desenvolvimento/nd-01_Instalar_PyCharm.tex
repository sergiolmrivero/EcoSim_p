\documentclass[10pt,a4paper]{article}
\usepackage[utf8]{inputenc}
\usepackage{amsmath}
\usepackage{amsfonts}
\usepackage{amssymb}
\usepackage{graphicx}
\usepackage{url}
\usepackage{hyperref}
\author{Sérgio Rivero}
\title{Nota de Desenvolvimento 01 \\ 
		Instalando e utilizando o Pycharm}
\begin{document}
\maketitle

\section{Introdução}

Pessoal. Fiz uma avaliação rápida das IDEs para Python. Como não faz sentido forcá-los a aprender Emacs, a solução ideal é uma IDE que seja eficiente para desenvolvimento, gratuita e rode bem em windows e linux.

Avaliei três : Spyder, Visual Studio Code e PyCharm.

Das três, a que melhor atende os requisitos e com o maior conjunto de funcionalidades é o PyCharm.

Os links que aparecem destacados neste texto são todos usáveis (é só clicar e ir para o site)

O programa pode ser encontrado em \url{https://www.jetbrains.com/pycharm/}

\section{Download e Instalação}

Há duas versões para download: Profissional e Comunidade. A comunidade é de livre uso.

É só clicar no botão \href{https://www.jetbrains.com/pycharm/download/download-thanks.html?platform=windows&code=PCC}{Downloads} que baixa.

No ubuntu o comando é:

\begin{verbatim}
sudo snap install pycharm-community --classic
\end{verbatim}


\section{Aprendendo a usar o PyCharm}

A primeira coisa é dar uma olhada nos vídeos instrucionais do PyCharm.

Estes vídeos estão nas webpages da companhia que faz o software.

Eles podem ser encontrados aqui:

\url{https://www.jetbrains.com/pycharm/learning-center/}

Há uma série de vídeos lá:

Olhem os 3 primeiros:

\begin{enumerate}
	\item \href{https://www.youtube.com/watch?v=BPC-bGdBSM8}{Getting Started with PyCharm 1/8: Setup}
	
	\item \href{https://www.youtube.com/watch?v=wCJ5kiSmvUY}{Getting Started with PyCharm 2/8: PyCharm UI and Projects}
	
	\item \href{https://www.youtube.com/watch?v=JLfd9LOdu_U}{Getting Started with PyCharm 3/8: Running Python Code}
\end{enumerate}



\end{document}